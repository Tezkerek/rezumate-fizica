\section{Ipoteza de Broglie. Difracția electronilor. Aplicații}

Analog cu dualismul undă-corpuscul în cazul undelor electromagnetice, Louis de Broglie asociază oricărei microparticule în mișcare cu energia $E$ și impulsul $p$ o undă caracterizată prin frecvența $\nu$ și lungimea de undă $\lambda$, cu relațiile dintre mărimi
\[
    E = h \nu
    \begin{Bmatrix}
        \nu & \lambda \\
        E   & p
    \end{Bmatrix}
    p = \frac{h}{\lambda}
\]

De Broglie a presupus că lungimea de undă a undelor asociate microparticulelor
trebuie să fie dată tot de relația \( \lambda = \frac{h}{p} \), unde $p$ este
impulsul microparticulei.

\parbreak
Ipoteza de Broglie afirmă că oricărei microparticule care posedă un impuls $p$
i se poate asocia în mod formal o undă cu lungimea de undă
\( \lambda_B = \frac{h}{p} \), numită lungime de undă de Broglie.

Undelor electromagnetice le sunt asociați fotonii, care nu au masă de repaus.
Analog, undele de Broglie sunt asociate particulelor cu masă de repaus:
electroni, protoni, neutroni, particule \alpha, molecule de hidrogen. În
concluzie, radiația electromagnetică are proprietăți ondulatorii și
corpusculare, asemenea radiației corpusculare.

\subsection*{Experimentul Davisson-Germer}

Fizicienii Davisson și Germer au confirmat experimental ipoteza de Broglie,
demonstrând că electronii în mișcare prezintă proprietăți ondulatorii, prin
generarea fenomenelor de difracție însoțite de interferență.

Filamentul $F$, alimentat de sursa $E_1$, emite electroni care sunt accelerați
într-un tun electronic, alimentat de sursa $E_2$. Tensiunea de accelerare $U$
este controlată de reostatul $R$ și măsurată cu voltmetrul $V$.
