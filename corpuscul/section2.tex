\section{Ipoteza de Broglie. Difracția electronilor. Aplicații}

Analog cu dualismul undă-corpuscul în cazul undelor electromagnetice, Louis de Broglie asociază oricărei microparticule în mișcare cu energia $E$ și impulsul $p$ o undă caracterizată prin frecvența $\nu$ și lungimea de undă $\lambda$, cu relațiile dintre mărimi
\[
    E = h \nu
    \begin{Bmatrix}
        \nu & \lambda \\
        E   & p
    \end{Bmatrix}
    p = \frac{h}{\lambda}
\]

De Broglie a presupus că lungimea de undă a undelor asociate microparticulelor
trebuie să fie dată tot de relația \( \lambda = \frac{h}{p} \), unde $p$ este
impulsul microparticulei.
Ipoteza de Broglie afirmă că oricărei microparticule care posedă un impuls $p$
i se poate asocia în mod formal o undă cu lungimea de undă
\( \lambda_B = \frac{h}{p} \), numită lungime de undă de Broglie.
