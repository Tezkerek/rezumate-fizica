\section{Dualismul undă-corpuscul}

Fenomenele de interferență, difracție, și polarizare evidențiază proprietățile
ondulatorii ale luminii, pe când efectul fotoelectric și efectul Compton
evidențiază proprietățile corpusculare. Prin urmare, se constată caracterul
dual, ondulatoriu, și corpuscular al undelor electromagnetice.

Efectul fotoelectric și efectul Compton au fost explicate considerând că
radiația electromagnetică este un flux de particule, numite fotoni, ce
interacționează cu substanța.

Mărimile ce caracterizează un foton sunt:
\begin{itemize}
    \newcommand{\boxlabel}[1]{\makebox[4cm][l]{#1}}
    \item \boxlabel{sarcina electrică} $q = 0$
    \item \boxlabel{masa de repaus} $m_0 = 0$
    \item \boxlabel{energia} $E = h\nu$
    \item \boxlabel{masa de mișcare} \( m = \frac{h\nu}{c^2} \)
    \item \boxlabel{impulsul} \( p = \frac{h\nu}{c} = \frac{h}{\lambda} \)
\end{itemize}

Constanta lui Planck, $h$, reprezintă veriga de legătură dintre aspectul
ondulatoriu și cel corpuscular. Poate fi exprimată prin produsul a două mărimi,
una ce caracterizează unda (frecvența $\nu$, perioada $T = \frac{1}{\nu}$,
lungimea de undă $\lambda$) și una caracteristică particulei (energia $E$, impulsul $p$).

\[ h = ET = p\lambda \]

În cazul radiațiilor X și $\gamma$ predomină caracterul corpuscular, energia și
impulsul fiind mari, iar perioada și lungimea de undă mici. În cazul undelor
radio predomină caracterul ondulatoriu, energia și impulsul fiind mici, iar
perioada și lungimea de undă mari.

În sistemele macroscopice, constanta lui Planck poate fi considerată nulă.
