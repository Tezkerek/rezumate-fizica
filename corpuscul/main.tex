%! TEX program = lualatex
\documentclass[a4paper, 12pt]{article}

\usepackage[margin=2.5cm]{geometry}
\usepackage{microtype}
\usepackage{indentfirst}
\usepackage{polyglossia}
\usepackage[colorlinks=true, allcolors=black]{hyperref}
\usepackage{graphicx}
\usepackage[font=footnotesize, labelfont=bf]{caption}
\usepackage{wrapfig}
\usepackage{amsmath}
\usepackage{mathrsfs}
\usepackage{icomma}
\usepackage{titling}

\setdefaultlanguage{romanian}

\addto\captionsromanian{
    \renewcommand{\figurename}{Fig.}
}

% Enable usage of \\ in sections
\pdfstringdefDisableCommands{%
    \def\\{}%
}

\newcommand{\mktitle}{%
    \noindent
    {\small\theauthor}

    \begin{center}
        \LARGE\thetitle
    \end{center}
}
\newcommand{\parbreak}{\vspace{1cm}}
\newcommand{\lorentzradical}{\ensuremath{\sqrt{1 - \frac{V^2}{c^2}}}}
\newcommand{\dd}{\mathrm{d}}
\newcommand{\diff}[2]{\frac{\mathrm{d}#1}{\mathrm{d}#2}}
\newcommand{\accentcolor}{magenta}

\title{Dualismul undă-corpuscul. Ipoteza de Broglie}
\author{Andrei Ancuța \\ clasa a XII-a B \\
Liceul Teoretic de Informatică „Grigore Moisil”, Iași}

\begin{document}

\mktitle
\tableofcontents

\section{Spectre}

Spectrul reprezintă un ansamblu discret sau continuu de valori care pot fi
luate de o anumită mărime. În particular, componentele monocromatice ale unei
radiații electromagnetice.

Spectroscopul este un instrument care descompune radiația electromagnetică
complexă în componentele monocromatice, utilizat pentru studiul spectrelor.

\subsection{Spectroscopul cu prismă}

În componența unui aparat spectral intră un dispozitiv ce descompune radiația
emisă în radiațiile monocromatice componente.

În spectroscopul cu prismă, această sarcină o are prisma optică, pe baza
fenomenului de dispersie.

\begin{wrapfigure}{r}{0.6\textwidth}
    \includegraphics[width=0.6\textwidth]{fig/spectroscop}
    \caption{Schema spectroscopului}
\end{wrapfigure}

Spectroscopul cu prismă are în componență un tub $C$, numit \emph{colimator},
prevăzut cu o fantă $F$ aflată în focarul unui sistem optic convergent $L_1$.
Sursa $S$ emite raze divergente, ce trec prin $F$ și sunt transformate într-un
fascicul paralel de către $L_1$.

Prisma optică $P$ va dispersa radiațiile incidente sub diferite unghiuri,
fasciculele urmând să fie focalizate de obiectivul $L_2$ de la capătul lunetei
$L$, în planul focal al obiectivului.

\clearpage

În ocularul lunetei intră și fasciculul de lumină reflectat pe fața prismei,
provenit de la un alt colimator, ce proiectează imaginea unei riglete
micrometrice $s$, iluminată de becul $B$.

Imaginile fantei corespunzătoare radiații monocromatice reprezintă liniile
spectrale observate prin intermediul ocularului, ce are rol de lupă.

Dacă în planul focal al obiectivului $L_2$ se așază o placă fotografică,
aparatul spectral poartă denumirea de \emph{spectograf}.

Într-un aparat spectral, prisma și lentilele sunt din sticlă pentru domeniul
vizibil, din cuarț pentru ultraviolet, și din materiale transparente, precum
monocristalul din clorură de sodiu, pentru infraroșu.

Spectrele pot fi: \emph{continue} sau \emph{discontinue}.

Setul de lungimi de undă de diferite lungimi de undă se împarte în
\emph{spectre de emisie} și \emph{spectre de absorbție}.

Spectrele de emisie caracterizează substanța emițătoare de lumină, iar cele de
absorbție, substanță absorbantă.

În funcție de natura sursei emitente, spectrele de emisie pot fi continue, sau discontinue de linii sau de bandă.

În funcție de sistemul de particule studiat, spectrele pot fi atomice sau moleculare.

\subsubsection{Spectre continue de emisie}

Lumina emisă de toate corpurile solide și lichide incandescente, precum
filamentul unui bec, un cărbune, sau metalul topit, conține o multitudine de
radiații spectrale suprapuse, cu lungimi de undă foarte apropiate.

Corpurile solide sau lichide, aduse la incandescență, emit un spectru continuu de culori: roșu, portocaliu, galben, verde, albastru, indigo, violet.

De exemplu, filamentul de wolfram al unui bec emite o lumină de culoare
diferită în funcție de temperatură. La 700$^\circ$C emite roșu închis, la
1000$^\circ$C roșu aprins, la 1200$^\circ$C portocaliu, la 1300$^\circ$C alb,
și la 1400$^\circ$C alb strălucitor. La fel și în cazul stelelor, clasificate
în funcție de culoare, care depinde de temperatură.

\subsubsection{Spectre discontinue de emisie}

Spectrele discontinue sunt emise de gazele din tuburile de descărcare (vapori
de mercur, sodiu, potasiu). Acestea pot fi spectre de linii sau spectre de
bandã.

Spectrele de linii sunt emise de substanțele gazoase aflate în stare atomică,
și iau forma unor linii strălucitoare de diferite culori, pe un fond negru.

Spectrele de bandă sunt asemănătoare spectrelor de linii, însă liniile sunt
grupate în benzi. Sunt emise de substanțele gazoase aflate în stare moleculară
($H_2$, $O_2$, $N_2$).

Înmuind un fir de platină într-o soluție NaCl și introducându-l în flacăra unui
bec Bunsen, flacăra va lua aspectul caracteristic fiecărui metal: galben pentru
natriu, roșu purpuriu pentru potasiu etc.

\subsubsection{Spectre de absorbție}

Dacă un spectru continuu traversează un gaz la o temperatură mai joasă decât
temperatura sursei, o soluție lichidă, sau o sticlă colorată, se obține un
spectru de absorbție. Acesta se reprezintă ca un spectru continuu brăzdat de
lini sau benzi întunecate.

Mărimile ce caracterizează un foton sunt:
\begin{itemize}
    \newcommand{\boxlabel}[1]{\makebox[4cm][l]{#1}}
    \item \boxlabel{sarcina electrică} $q = 0$
    \item \boxlabel{masa de repaus} $m_0 = 0$
    \item \boxlabel{energia} $E = h\nu$
    \item \boxlabel{masa de mișcare} \( m = \frac{h\nu}{c^2} \)
    \item \boxlabel{impulsul} \( p = \frac{h\nu}{c} = \frac{h}{\lambda} \)
\end{itemize}

Constanta lui Planck, $h$, reprezintă veriga de legătură dintre aspectul
ondulatoriu și cel corpuscular. Poate fi exprimată prin produsul a două mărimi,
una ce caracterizează unda (frecvența $\nu$, perioada $T = \frac{1}{\nu}$,
lungimea de undă $\lambda$) și una caracteristică particulei (energia $E$, impulsul $p$).

\[ h = ET = p\lambda \]

În cazul radiațiilor X și $\gamma$ predomină caracterul corpuscular, energia și
impulsul fiind mari, iar perioada și lungimea de undă mici. În cazul undelor
radio predomină caracterul ondulatoriu, energia și impulsul fiind mici, iar
perioada și lungimea de undă mari.

În sistemele macroscopice, constanta lui Planck poate fi considerată nulă.

\section{Efectul Compton}

Atunci când un fascicul de raze X, provenit de la o sursă $S$, trece printr-un
bloc de grafit $G$, radiațiile incidente sunt împrăștiate în toate direcțiile.

\begin{wrapfigure}[9]{r}{0.5\textwidth}
    \includegraphics[width=0.5\textwidth]{fig/compton}
    \caption{Experimentul Compton, reprezentat schematic}
\end{wrapfigure}

Pentru diferite unghiuri de împrăștiere $\theta$, detectorul $D$ înregistrează,
pe lângă radiația incidentă cu lungimea de undă $\lambda_0$, și o altă radiație
cu lungimea de undă \( \lambda > \lambda_0 \).

Din punct de vedere macroscopic, lumina, și în general radiația electromagnetică,
este o undă. Din punct de vedere microscopic, lumina este un ansamblu de particule
cuantice.

\parbreak

Fenomenul, observabil pentru lungimi de undă mici, ca în cazul razelor X și
\gamma, deci pentru frecvențe mari (\(\lambda = \frac{c}{\nu} \)), a fost
explicat de către Compton pe baza naturii corpusculare a undelor electromagnetice,
adică prin existența fotonilor.

\emph{%
    Efectul Compton este fenomenul de împrăștiere elastică a fotonilor pe
    electronii liberi, în urma căreia, pe lângă radiația incidentă, apare și
    o radiație cu lungimea de undă mai mare (frecvența mai mică).
}

În cazul în care atomii substanței pe care se produce împrăștierea sunt ușori,
ca în cazul atomilor de siliciu, bor sau bariu, atunci energia de legătură a
electronilor de valență este mult mai mică decât energia fotonului incident
$h\nu_0$, iar electronul poate fi considerat practic liber.


\clearpage

\section*{Bibliografie}
\begin{itemize}
    \item Manualul de fizică pentru clasa a XII-a, F1 \\
        Cleopatra Gherbanovschi, Nicolae Gherbanovschi \\
        Editura NICULESCU ABC \\
        2016
\end{itemize}

\end{document}
