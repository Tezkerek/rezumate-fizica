\section{Modelul Bohr}

Ideea corectă a modelului Rutherford de existență a unui nucleu atomic în care
este concentrată aproape toată masa și toată sarcina pozitivă a atomului a fost
preluată în modelele atomice propuse ulterior.

În conceptul lui Bohr, atomul este un sistem solar în miniatură, cu forțe
electrice în loc de forțe gravitaționale. Nucleul încărcat pozitiv joacă rolul
soarelui, iar electronii se mișcă în jurul lui sub acțiunea forțelor electrice
de atracție.

Acest model atomic se bazează pe \emph{postulatele lui Bohr}.

În fizica clasică, energia unui sistem poate varia în mod continuu.

Având o masă mai mică decât a atomilor, electronii nu vor pierde energie la o
ciocnire, ci vor fi \emph{împrăștiați elastic}. Însă în urma unei ciocniri
inelastice cu un atom, electronul va pierde energie, iar viteza lui va scădea.
Energia pierdută va fi transferată atomului ciocnit, care trece într-o
\emph{stare excitată}. Se constată, pe cale experimentală, că atomul poate
primi energie numai în anumite valori bine determinate.
\emph{Atomul poate avea numai anumite stări (discrete) de energie}.
