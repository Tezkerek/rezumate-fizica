\section{Compunerea vitezelor}

Vom exprima legile relativiste de transformare a vitezelor corpurilor dintr-un
SRI în altul, pornind de la transformările Lorentz deduse anterior:
\begin{equation*}
    \begin{aligned}[t]
        x = \frac{x' + Vt'}{\lorentzradical}
    \end{aligned}\qquad\qquad
    \begin{aligned}[t]
        y = y'
    \end{aligned}\qquad\qquad
    \begin{aligned}[t]
        z = z'
    \end{aligned}\qquad\qquad
    \begin{aligned}[t]
        t = \frac{t' + \frac{Vx'}{c^2}}{\lorentzradical}
    \end{aligned}
\end{equation*}

Din relațiile de mai sus rezultă:
\begin{equation*}
    \begin{aligned}[t]
        \dd x = \frac{\dd x' + V\dd t'}{\lorentzradical}
    \end{aligned}\qquad\qquad
    \begin{aligned}[t]
        \dd y = \dd y'
    \end{aligned}\qquad\qquad
    \begin{aligned}[t]
        \dd z = \dd z'
    \end{aligned}\qquad\qquad
    \begin{aligned}[t]
        \dd t = \frac{\dd t' + \frac{V\dd x'}{c^2}}{\lorentzradical}
    \end{aligned}
\end{equation*}

Împărțim primele trei ecuații la cea de-a patra și obținem:
\begin{equation*}
    \begin{aligned}[t]
        \dv{x}{t} = \frac{\dd x' + V\dd t'}{\dd t' + \frac{V\dd x'}{c^2}}
    \end{aligned}\qquad\qquad
    \begin{aligned}[t]
        \dv{y}{t} = \frac{\dd y'\lorentzradical}{\dd t' + \frac{V\dd x'}{c^2}}
    \end{aligned}\qquad\qquad
    \begin{aligned}[t]
        \dv{z}{t} = \frac{\dd z'\lorentzradical}{\dd t' + \frac{V\dd x'}{c^2}}
    \end{aligned}
\end{equation*}
sau:
\begin{equation*}
    \begin{aligned}[t]
        \dv{x}{t} = \frac{\dv{x'}{t'} + V}{1 + \frac{V}{c^2}\dv{x'}{t'}}
    \end{aligned}\qquad\qquad
    \begin{aligned}[t]
        \dv{y}{t} = \frac{\dv{y'}{t'}\lorentzradical}{1 + \frac{V}{c^2}\dv{x'}{t'}}
    \end{aligned}\qquad\qquad
    \begin{aligned}[t]
        \dv{z}{t} = \frac{\dv{z'}{t'}\lorentzradical}{1 + \frac{V}{c^2}\dv{x'}{t'}}
    \end{aligned}
\end{equation*}

Rezultă astfel formulele de compunere a vitezelor în teoria relativității
restrânse:
\begin{equation*}
    \color{\accentcolor}
    \begin{aligned}[t]
        v_x = \frac{v_x' + V}{1 + \frac{V}{c^2}v_x'}
    \end{aligned}\qquad\qquad
    \begin{aligned}[t]
        v_y = \frac{v_y'\lorentzradical}{1 + \frac{V}{c^2}v_x'}
    \end{aligned}\qquad\qquad
    \begin{aligned}[t]
        v_z = \frac{v_z'\lorentzradical}{1 + \frac{V}{c^2}v_x'}
    \end{aligned}
\end{equation*}

Pentru a obține formulele inverse, schimbăm accentele și înlocuim pe $V$ în
$-V$:
\begin{equation*}
    \begin{aligned}[t]
        v_x' = \frac{v_x - V}{1 - \frac{V}{c^2}v_x}
    \end{aligned}\qquad\qquad
    \begin{aligned}[t]
        v_y' = \frac{v_y\lorentzradical}{1 - \frac{V}{c^2}v_x}
    \end{aligned}\qquad\qquad
    \begin{aligned}[t]
        v_z' = \frac{v_z\lorentzradical}{1 - \frac{V}{c^2}v_x}
    \end{aligned}
\end{equation*}

Comparând formulele de mai sus cu cele clasice deduse din transformările
Galilei
\begin{equation*}
    \begin{aligned}[t]
        v_x = v_x' + V
    \end{aligned}\qquad\qquad
    \begin{aligned}[t]
        v_y = v_y'
    \end{aligned}\qquad\qquad
    \begin{aligned}[t]
        v_z = v_z'
    \end{aligned}
\end{equation*}
se observă apariția numitorului \( \left( 1 + \frac{Vv_x'}{c^2} \right) \) la
$v_x$, $v_y$ și $v_z$, și a radicalului Lorentz la numărătorul lui $v_y$ și al
lui $v_z$.

Pentru \( V \ll c \), avem \( \frac{V^2}{c^2} \ll 1 \) și astfel din
transformările Lorentz obținem transformările Galilei din fizica clasică.

Compunerea relativistă a vitezelor reconfirmă principiul conform căruia viteza
luminii în vid $c$ este viteza maximă, și nu poate fi atinsă de corpuri și
particule. Pentru \( V = c \), se obține:
\begin{equation*}
    \begin{aligned}
        v_x &= \frac{v_x' + c}{1 + \frac{cv_x'}{c^2}} = c \\
        v_x' &= \frac{v_x - c}{1 - \frac{cv_x}{c^2}} = -c
    \end{aligned}
\end{equation*}

Un alt exemplu ar fi \( v_x' = c \). Conform formulelor clasice ale lui Galilei, am obține:
\[ v_x = c + V > c \]
Însă din transformările Lorentz ne rezultă:
\[ v_x = \frac{c + V}{1 + \frac{Vc}{c^2}} = \frac{c + V}{c + V} \cdot c = c \]

Dacă \( v_x' = c \) și \( V = c \), ar rezulta \( v_x = c + c = 2c \), ceea ce
contrazice experiențele lui Michelson și Morley.

\pagebreak

\subsubsection*{Problemă rezolvată}

O navă cosmică, îndepărtându-se de Pământ cu viteza de $0,9c$, lansează un
vehicul cosmic pe aceeași direcție cu mișcarea ei. Viteza vehiculului este
$0,9c$ față de navă ($c = 2,99792 \cdot 10^8$  m/s). Care este viteza
vehiculului cosmic în raport cu Pământul?

\parbreak

Considerăm $S$ referențialul legat de Pământ, $S'$ referențialul legat de navă,
$v$ viteza vehiculului față de Pământ, $v'$ viteza vehiculului față de navă, și
$V$ viteza navei față de Pământ. Rezultă că legea de compunere relativistă a
vitezelor are forma:
\[ v = \frac{v' + V}{1 + \frac{Vv'}{c^2}} \]

Calculând, obținem:
\[ v = \frac{0,9c + 0,9c}{1 + \frac{0,9^2 c^2}{c^2}} = 0,994c = 297993 ~ \mathrm{km/s} \]

Dacă am fi folosit transformările Galilei pentru calculul nerelativist al
vitezei, am fi obținut o viteză a vehiculului față de Pământ imposibilă, egală
cu $1,8c$.

